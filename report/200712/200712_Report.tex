%%%%%%%%%%%%%%%%%%%%%%%%%%%%%%%%%%%%%%%%%
% University/School Laboratory Report
% LaTeX Template
% Version 3.1 (25/3/14)
%
% This template has been downloaded from:
% http://www.LaTeXTemplates.com
%
% Original author:
% Linux and Unix Users Group at Virginia Tech Wiki 
% (https://vtluug.org/wiki/Example_LaTeX_chem_lab_report)
%
% License:
% CC BY-NC-SA 3.0 (http://creativecommons.org/licenses/by-nc-sa/3.0/)
%
%%%%%%%%%%%%%%%%%%%%%%%%%%%%%%%%%%%%%%%%%

%----------------------------------------------------------------------------------------
%	PACKAGES AND DOCUMENT CONFIGURATIONS
%----------------------------------------------------------------------------------------

\documentclass{article}

\usepackage[version=3]{mhchem} % Package for chemical equation typesetting
\usepackage{siunitx} % Provides the \SI{}{} and \si{} command for typesetting SI units
\usepackage{graphicx} % Required for the inclusion of images
%\usepackage{natbib} % Required to change bibliography style to APA
\usepackage{amsmath} % Required for some math elements 
\usepackage{csquotes}

%\usepackage[urlcolor=red]{hyperref}
\usepackage{xcolor}

\usepackage[colorlinks = true,
            linkcolor = blue,
            urlcolor  = blue,
            citecolor = blue,
            anchorcolor = blue]{hyperref}

\setlength\parindent{0pt} % Removes all indentation from paragraphs

\renewcommand{\labelenumi}{\alph{enumi}.} % Make numbering in the enumerate environment by letter rather than number (e.g. section 6)

\renewcommand\UrlFont{\color{blue}\rmfamily}

%\usepackage{times} % Uncomment to use the Times New Roman font

%----------------------------------------------------------------------------------------
%	DOCUMENT INFORMATION
%----------------------------------------------------------------------------------------

\title{\textbf{REPORT} \\ 06.07.2020 - 12.07.2020} % Title

%\author{John \textsc{Smith}} % Author name

%\date{\today} % Date for the report

\date{}

\begin{document}

\maketitle % Insert the title, author and date

%\begin{center}
%\begin{tabular}{l r}
%Date Performed: & January 1, 2012 \\ % Date the experiment was performed
%Partners: & James Smith \\ % Partner names
%& Mary Smith \\
%Instructor: & Professor Smith % Instructor/supervisor
%\end{tabular}
%\end{center}

% If you wish to include an abstract, uncomment the lines below
% \begin{abstract}
% Abstract text
% \end{abstract}

%----------------------------------------------------------------------------------------
%	SECTION 1
%----------------------------------------------------------------------------------------


\section*{Summary}
Since the code structure of Shift-GCN \cite{Cheng_2020_CVPR} is very similar with 2S-AGCN \cite{Shi_2019_CVPR} which has been added \enquote{bone weights} feature. Both of them process \enquote{joint} and \enquote{bone} separately and ensemble the two information together. 


Unfortunately, both of them give no good performance for Calo's data: both of them fluctuate about 32\% accuracy after 50 epochs, and it not seems that it can be improved after some more epochs. Compared them two, there is no evident improvement for Calo's data. \\

The Semantics-Guided \cite{Zhang_2020_CVPR} model can reach about 90.01\% accuracy with learnable parameters for Calo's data after 100 epochs with 663963 learnable parameters. 

Since Semantics-Guided \cite{Zhang_2020_CVPR} performs well for Calo's data, using it as base code model, extracted core component of \enquote{Shift} feature of Shift-GCN \cite{Cheng_2020_CVPR} and integrated it with Semantics-Guided \cite{Zhang_2020_CVPR} with 175312 learnable parameters, the accuracy reaches about 85.47\%. \\

The integrated model compared with Semantics-Guided \cite{Zhang_2020_CVPR} model: it cannot surpass in term of accuracy, but it runs much faster than that.

The integrated model compared with Shift-GCN \cite{Cheng_2020_CVPR}: it runs much more faster and gets better accuracy.





%----------------------------------------------------------------------------------------
%	BIBLIOGRAPHY
%----------------------------------------------------------------------------------------

\bibliographystyle{ieeetr}

\bibliography{sample}

%----------------------------------------------------------------------------------------


\end{document}